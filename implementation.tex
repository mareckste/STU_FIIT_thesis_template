\newpage
\section{Implementácia}
\lipsum[1]

\subsection{Podsekcia 1} \label{iclass}
\noindent
\lipsum[1]

\begin{algorithm}[H]
	\small
	\caption{Načítanie údajov}\label{kinect}
	\algorithmicrequire $depthMat$, $colorMat$\\
	\algorithmicensure $depthMat$, $colorMat$
	
	\begin{algorithmic}[1]
		\Procedure{GetMappedColorData}{}
		\If {\textbf{getFrame}($frame$) = $success$}
		\State \textbf{mapData}($frame$, $mapped$)
		\If {\textbf{getColor}($frame$, $colorFrame$) = $success$}
		\State \textbf{copyFrameData}($colorFrame$, $colorData$)
		\For {$i<$\textbf{width}($color$)}
		\For {$j<$\textbf{height}($color$)}
		\State $depthSpacePoint \gets mapped[i][j]$
		\If {\textbf{visible}($depthSpacePoint$) $\not= success$}
		\State $depthMat[i][j]\gets -1$
		\Else
		\State $index\gets$ \textbf{getIndex}($depthSpacePoint$)
		\State $depthMat[i][j]\gets depthData[index]$
		\EndIf
		\State $colorMat[i][j]\gets colorData[i][j]$
		\EndFor
		\EndFor
		\EndIf
		\EndIf
		\EndProcedure
	\end{algorithmic}
\end{algorithm}
\noindent
V prvom pseudokóde so vstupnými a výstupnými parametrami typu \textbf{Mat} (typ reprezentujúci maticu definovaný v OpenCV) sme použili následovné metódy: \dots